% Kommentera bort de punkter som inte behövs.

\section*{Projekt}
\textit{Vilket projekt har gruppen valt? Om eget projekt, ska detta beskrivas.}

\section*{Mål}
\textit{Det gemensamma målet med gruppens arbete.}

\section*{Deadlines}
\textit{Vilka deadlines finns för gruppen i kursen, vilka egna deadlines/milstolpar sätter gruppen upp. Kanske även de individuella uppgifternas deadlines ska finnas med här för att lättare kunna ta hänsyn till dessa vid planering?}

Projektrapporten ska vara inskickad på blackboard senast 21 maj 2018 kl 12:00.

\section*{Tidsåtgång}
\textit{Hur ofta ska vi träffas? Hur långa träffar ska vi ha?}

\section*{Atmosfär}
\textit{Hur vill vi behandla varandra – vilka trivselregler ska gälla i vår grupp?}

\section*{Personligheter}
\textit{Hur hanterar vi olikheter – en del pratar mycket, andra mindre. En del gillar detaljer, andra fokuserar hellre på helheter. \\
Måste alla prata lika mycket? \\
Måste alla göra lika mycket? \\
Vad är lika mycket?}

\section*{Ansvar}
\textit{Vad förväntar vi oss av varandra? \\
\indent - inför en träff, förberedelser mm?}

\section*{Arbetsfördelning}
\textit{Hur arbetar vi i gruppen? Vem gör vad? Hur bestämmer vi det? Hur fattar vi beslut?}

\section*{Kommunikation}
\textit{Hur kommunicerar vi med varandra i gruppen före, mellan och efter mötena? Kommunikation via mail? Mobil? Facebook-grupp/sida? Skype etc.}

Kommunikation mellan gruppmedlemmarna sker på Discord.

\section*{Feedback}
\textit{Ska vi i gruppen ge varandra positiv feedback/kritik? Hur ska det i så fall gå till?}

\section*{Dokumentation}
\textit{Ska det vi kommer överens om skrivas ner – inte bara ha det muntligt? Hur ska det i så fall gå till? Vem ska skriva? \\
Ska vi ta närvaro på alla möten?}

När gruppen har möte så skrivs det ner vad som har sagts på mötet och vilka som var närvarande. Alla dokument som skrivs sparas på \href{https://github.com/anhility/DVA227}{samarbetsytan}.

\section*{Roller}
\textit{Ska vi ha uttalade roller i gruppen – vilka roller och ska de i så fall rotera?}

Alla medlemmar i gruppen kommer vara projektledare någon gång, detta roterar med 3 veckors intervaller.

\section*{Närvaro}
\textit{Hur och vem kontaktar jag om jag inte tänker komma på ett möte? \\
Vad är giltig frånvaro?}

\section*{Återkoppling}
\textit{Vem kontaktar den som inte deltagit på ett möte så att den får information om vad som hänt på mötet?}

Om en medlemm inte kunde delta på ett möte så kan medlemmen gå in på \href{https://github.com/anhility/DVA227}{samarbetsytan} och läsa protokollet för mötet. 

\section*{Konsekvenser}
\textit{Vilka konsekvenser blir det för den som inte deltagit på ett möte eller gjort det den ska ha gjort? Här kanske gruppen behöver sätta upp en stege av konsekvenser.}

\section*{Konflikter}
\textit{Hur hanterar vi oenigheter/konflikter i gruppen?}

\textbf{Vid konflikt i gruppen som inte kan lösas av gruppmedlemmarna själva skall examinator kontaktas som sedan tar beslut för gruppen. \\ \\
Ingen gruppmedlem får uteslutas ur gruppen eller utelämnas från rapporten utan beslut från examinator.}

\section*{Ambitionsnivå}
\textit{Hur ser det ut – vill alla nå samma resultat? Hur hanterar vi olika ambitionsnivåer?}

Gruppen siktar mot att ligga på nivå som motsvarar betyg 4.

\section*{Kontraktsbrott}
\textit{Hur och vad gör vi om någon i gruppen bryter mot det vi kommit fram till i kontraktet?}
