% Kommentera bort de punkter som inte behövs.

\section*{Närvarande}
\begin{itemize}[noitemsep]
    \item Jocke
    \item Sara
    \item Micke \textit{- sekreterare}
\end{itemize}

\section*{Genomgång av föregående mötesprotokoll}
Inget tidigare mötesprotokoll.

\section*{Var har gjorts sedan föregående möte?}
Inget tidigare möte men detta har hänt:
\begin{itemize}[noitemsep]
    \item Val av projektledarordning (Micke, Isak, Ville).
    \item Dokumentationsmallar och workflow har skapats och definierats.
    \item Samarbetsyta har skapats: \url{http://github.com/anhility/DVA227}
    \item Första version av projektspecifikation har skrivits.
\end{itemize}

\subsection*{Problem/Åtgärder?}
n/a

\section*{Vad ska göras innan nästa möte?}
\begin{itemize}[noitemsep]
    \item Förstudie klar senast v13.
    \begin{itemize}[noitemsep]
        \item MoSCoW
        \item QTR-triangeln
        \item WBS - förstudie
        \item Logisk nätplan - förstudie
    \end{itemize}
    \item Skriva klar projektstrukturen.
    \item Planera planeringsfasen.
    \item Tid och resursplanering.
    \item Starta planeringsfasen v13
    \begin{itemize}[noitemsep]
        \item Detaljerad WBS
        \item Kritiska linjen
        \item Tidsplan \& Gantt-schema
        \item Resurshantering
        \item Kostnadskalkyl \& budget
        \item Riskhantering
        \item Produktkvalitet
    \end{itemize}
    \item Skriva klart gruppkontrakt innan gruppmöte v13.
    \item Gruppmöten varje vecka.
    \item Grupparbetstillfällen två gånger varje vecka ca 6 timmar var plus eget arbete.
    \item Planerat max 20 arbetstimmar per vecka per person.
    \item Inlämning av progressrapporter varje söndag till Jocke, Sara och Johnny.
    \item Möjligtvis en omdefiniering av projektspecifikationen när vi är mer insatta i projektet.
\end{itemize}

\subsection*{Problem/Åtgärder?}
n/a

\section*{Beslut}
tba

\section*{Uppföljning}
n/a