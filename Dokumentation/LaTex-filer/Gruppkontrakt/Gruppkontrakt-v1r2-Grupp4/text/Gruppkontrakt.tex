% Kommentera bort de punkter som inte behövs.

\section*{Projekt}
\textit{Vilket projekt har gruppen valt? Om eget projekt, ska detta beskrivas.}

En kund med ett globalt nätverk behöver en standardisering av hårdvara, mjukvara och IP-planering för alla sina under-siter för att ge bättre utbud för sina underkunder.

\section*{Mål}
\textit{Det gemensamma målet med gruppens arbete.}

Målet är att skapa ett lösningsförslag för kunden där alla under-siter följer en gemensam standardisering samt att all kommunikation mellan under-siter och huvudservrar är krypterad.

\section*{Deadlines}
\textit{Vilka deadlines finns för gruppen i kursen, vilka egna deadlines/milstolpar sätter gruppen upp. Kanske även de individuella uppgifternas deadlines ska finnas med här för att lättare kunna ta hänsyn till dessa vid planering?}

Projektrapporten ska vara inskickad på blackboard senast 21 maj 2018 kl 12:00.

\section*{Tidsåtgång}
\textit{Hur ofta ska vi träffas? Hur långa träffar ska vi ha?}

\begin{itemize}[noitemsep]
    \item Max 20 arbetstimmar per vecka per person.
    \item 2 st planerade grupparbetstillfällen per vecka.
    \item 1 planerat gruppmöte per vecka.
\end{itemize}

\section*{Atmosfär}
\textit{Hur vill vi behandla varandra – vilka trivselregler ska gälla i vår grupp?}

Generella trivselregler. Don't be a dick.

\section*{Personligheter}
\textit{Hur hanterar vi olikheter – en del pratar mycket, andra mindre. En del gillar detaljer, andra fokuserar hellre på helheter. \\
Måste alla prata lika mycket? \\
Måste alla göra lika mycket? \\
Vad är lika mycket?}

Alla får prata hur mycket eller lite de vill så länge alla får fram sina punkter ifall de vill tala.

Alla ska jobba med ungefär lika många delmoment samt ungefär lika många arbetstimmar totalt när projektet är slut.

\section*{Ansvar}
\textit{Vad förväntar vi oss av varandra? \\
\indent - inför en träff, förberedelser mm?}

\begin{itemize}[noitemsep]
    \item Vara uppdaterad genom att läsa mötesprotokoll mm.
    \item \textbf{Projektmedlem:} Utföra de uppgifter du är tilldelad.
    \item \textbf{Projektledare: } Planera och delegera på ett genomförbart sätt.
\end{itemize}

\section*{Arbetsfördelning}
\textit{Hur arbetar vi i gruppen? Vem gör vad? Hur bestämmer vi det? Hur fattar vi beslut?}

Projektledaren bestämmer vad som skall göras. Officiella beslut fattas på gruppmöten. Arbete under planerade grupparbetstillfällen ska prioriteras av gruppmedlemmar. Individuellt arbete prioriteras i mån av tid med hänsyn till parallella kurser. Målet är max 20 arbetstimmar per vecka per person.

\section*{Kommunikation}
\textit{Hur kommunicerar vi med varandra i gruppen före, mellan och efter mötena? Kommunikation via mail? Mobil? Facebook-grupp/sida? Skype etc.}

Kommunikation mellan gruppmedlemmarna sker på Discord, email och telefon/sms.

\section*{Feedback}
\textit{Ska vi i gruppen ge varandra positiv feedback/kritik? Hur ska det i så fall gå till?}

Casual feedback/kritik mellan gruppmedlemmar.

Officiell feedback/kritik på gruppmöten.

\section*{Dokumentation}
\textit{Ska det vi kommer överens om skrivas ner – inte bara ha det muntligt? Hur ska det i så fall gå till? Vem ska skriva? \\
Ska vi ta närvaro på alla möten?}

När gruppen har möte så skrivs det ner vad som har sagts på mötet och vilka som var närvarande. Alla dokument som skrivs sparas på \href{https://github.com/anhility/DVA227}{samarbetsytan}. Projektledaren är mötesordförande och gruppmedlemmarna roterar som sekreterare.

\section*{Roller}
\textit{Ska vi ha uttalade roller i gruppen – vilka roller och ska de i så fall rotera?}

Alla medlemmar i gruppen kommer vara projektledare någon gång, detta roterar med 3 veckors intervaller. Gruppmedlemmar som inte är projektledare roterar som sekreterare under gruppmöten.

\section*{Närvaro}
\textit{Hur och vem kontaktar jag om jag inte tänker komma på ett möte? \\
Vad är giltig frånvaro?}

Kontakta gruppen via discord, email eller sms. Exempel på giltig frånvaro: sjukdom, familjekris och liknande. Kontakta gruppen så fort som möjligt vid frånvaro.

\section*{Återkoppling}
\textit{Vem kontaktar den som inte deltagit på ett möte så att den får information om vad som hänt på mötet?}

Om en medlem inte kunde delta på ett möte så kan medlemmen gå in på \href{https://github.com/anhility/DVA227}{samarbetsytan} och läsa protokollet för mötet som var.

\section*{Konsekvenser}
\textit{Vilka konsekvenser blir det för den som inte deltagit på ett möte eller gjort det den ska ha gjort? Här kanske gruppen behöver sätta upp en stege av konsekvenser.}

Första gången får medlemmen en varning. Andra gången en varning och möjligtvis kontakt med lärarna om vad som har hänt. Tredje gången kontakt med lärarna direkt om möjlig åtgärd.

\section*{Konflikter}
\textit{Hur hanterar vi oenigheter/konflikter i gruppen?}

Vid konflikter försöker en överenskommelse hittas. Annars kan röstning under gruppmöten användas.

\textbf{Vid konflikt i gruppen som inte kan lösas av gruppmedlemmarna själva skall examinator kontaktas som sedan tar beslut för gruppen. \\
Ingen gruppmedlem får uteslutas ur gruppen eller utelämnas från rapporten utan beslut från examinator.}

\section*{Ambitionsnivå}
\textit{Hur ser det ut – vill alla nå samma resultat? Hur hanterar vi olika ambitionsnivåer?}

Gruppen siktar mot att ligga på nivå som motsvarar betyg 4.

\section*{Kontraktsbrott}
\textit{Hur och vad gör vi om någon i gruppen bryter mot det vi kommit fram till i kontraktet?}

Tag upp problemet på gruppmöten.

Om det inte går att åtgärda måste problemet tas upp med examinator.

\newpage

\vspace*{\fill}

\noindent \textbf{Vi signerar att följa ovanstående punkter och regler:} \\ \\ \\

\noindent\begin{tabular}{ll}
    \makebox[6.5cm]{\hrulefill} & \\
    Mikael Andersson & \\[3em]
    \makebox[6.5cm]{\hrulefill} & \\
    Vilhelm Beijer & \\[3em]
    \makebox[6.5cm]{\hrulefill} & \makebox[6.5cm]{\hrulefill} \\
    Isak Söderström & Datum \\
\end{tabular}
