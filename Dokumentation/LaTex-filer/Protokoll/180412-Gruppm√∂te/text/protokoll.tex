% Kommentera bort de punkter som inte behövs.

\section*{Närvarande}
\begin{itemize}[noitemsep]
    \item Mikael
    \item Isak \textit{- sekreterare}
    \item Vilhelm
    \item Sara
    \item Jocke
\end{itemize}

\section*{Agenda}
\begin{itemize}
	\item Projektet har skalats ner och omdefinierats och måste kollas med lärare om det är en rimlig storlek jämfört med det tidigare projektet.
\end{itemize}

\section*{Genomgång av föregående mötesprotokoll}
\begin{itemize}[noitemsep]
    \item 180406-Gruppmöte.pdf
\end{itemize}

\section*{Var har gjorts sedan föregående möte?}
\begin{itemize}[noitemsep]
    \item Planering
    \begin{itemize}[noitemsep]
        \item Fastställt den logiska nätplanen
        \item Gjort en Aktivitetslista
        \item Gjort ett Gantt schema
        \item Riskanalys
    \end{itemize}
\end{itemize}

\subsection*{Problem/Åtgärder?}

\section*{Vad ska göras innan nästa möte?}
\begin{itemize}[noitemsep]
	\item Börja med arbetspaketen i projektet.
	\begin{itemize}
		\item 1.1.5 Minimikrav
		\item 3.1 Start av rapport
		\item 1.1.1 Routing
	\end{itemize}
\end{itemize}
\subsection*{Problem/Åtgärder?}
\begin{itemize}
	\item Inte fått svar av Johnny på de frågor vi behöver veta för att kunna göra klart några av de första arbetspaketen.
	\begin{itemize}
		\item Börja med de saker vi kan göra, t.ex. start av rapport.
	\end{itemize}
\end{itemize}

\section*{Beslut}
\begin{itemize}
	\item Jocke och Sara sa att den nya riktningen på projektet var okej så det är bara att köra igång.
\end{itemize} 
\section*{Uppföljning}

\newpage

\section*{Övrigt}
\begin{itemize}
	\item Vi behöver inte köra gruppmöte med Jocke och Sara vecka 16 om gruppen känner att det inte behövs.
\end{itemize}
