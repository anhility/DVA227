% Document type
	\documentclass[11pt, titlepage, a4paper]{article}
	% Packages
	\usepackage[utf8x]{inputenc}
	\usepackage[swedish]{babel}
	\usepackage{adjustbox}
	\usepackage{pdfpages}
	\usepackage{pdflscape}
    \usepackage{float,lscape}
    \usepackage{rotating}
	\usepackage{wrapfig}
    \usepackage{graphicx}
	\usepackage{amsmath}
	\usepackage{amsfonts}
	\usepackage{fancyhdr}
	\usepackage{enumerate}
	\usepackage{listings}
	\usepackage[titletoc]{appendix}
    \usepackage{caption}
	\usepackage[pdfborder={0 0 0},colorlinks=true, urlcolor=blue, citecolor=red, bookmarks=false]{hyperref}
	\usepackage[margin=1in,headheight=13.6pt]{geometry}
	\usepackage[absolute]{textpos}
	\usepackage[section]{placeins}
	\usepackage{url}
	\usepackage{tabularx}
%	\usepackage{gensymb}
	\usepackage{caption}
%	\usepackage{xltxtra}
	\usepackage{courier} %font
	% Page style
	\pagestyle{fancy}
	\marginparsep = 0pt
    \definecolor{bleudefrance}{rgb}{0.19, 0.55, 0.91}
	\definecolor{bittersweet}{rgb}{1.0, 0.44, 0.37}
	\definecolor{applegreen}{rgb}{0.55, 0.71, 0.0}
  	\definecolor{babyblueeyes}{rgb}{0.63, 0.79, 0.95}


	% Set font
	%\setromanfont{Calibri}



	


	\renewcommand\contentsname{Table of Contents}
	\newcommand{\HRule}{\rule{\linewidth}{0.5mm}}
	
	\newcommand{\circR}{\textsuperscript{\textregistered}}

	% Code style
	\lstset{
		backgroundcolor=\color[rgb]{0.92,0.92,0.92},
		basicstyle=\ttfamily\footnotesize,
		showspaces=false,
		showstringspaces=false,
		showtabs=false,
		tabsize=2,
		captionpos=b,
		breaklines=false,
		%keywordstyle=\color[rgb]{0,0,1},
		%commentstyle=\color[rgb]{0.133,0.545,0.133}
        %commentstyle=\color[rgb]{0,0,0}
	}

\begin{document}
\begin{titlepage}
% Title
\begin{center}
    \begin{figure}[t]
        \includegraphics[width=\tLogoWidth, viewport=0 0 100 100]{\tLogoLocation}
    \end{figure}

    \Large \tSchoolName \\
    \Large \tAcademyName \\
    \Large \tSchoolLocation \\

    \noindent\makebox[\linewidth]{\rule{\textwidth}{0.4pt}} \\ [0.5cm]

    \Large{\tCourseName} \\ [1.0cm]

%============================================================
% Title management

    \Huge \textbf{\underline{\tTitleOne}} \\ [0.5cm]
    \ifx
        \tTitleTwo \empty
            %nothing
        \else
            \huge \textbf{\uppercase{\tTitleTwo}} \\
    \fi
    \ifx
        \tTitleThree \empty
            ~\\ [0.5cm]
        \else
            \huge \textbf{\uppercase{\tTitleThree}} \\ [1cm]
    \fi

%============================================================
% Author management

    \ifx
        \tAuthorTitle \empty
            %nothing
        \else
            \LARGE \textbf{\underline{\smash{\tAuthorTitle}}} \\ [0.2cm]
    \fi
    
    
    \LARGE \tAuthorOneName \\
    \large \href{mailto:\tAuthorOneEmail}{\tAuthorOneEmail} \\
    
    \ifx
        \tAuthorTwoName \empty
            %nothing
        \else
            \LARGE \tAuthorTwoName \\
            \large \href{mailto:\tAuthorTwoEmail}{\tAuthorTwoEmail} \\
    \fi
    \ifx
        \tAuthorThreeName \empty
            %nothing
        \else
            \LARGE \tAuthorThreeName \\
            \large \href{mailto:\tAuthorThreeEmail}{\tAuthorThreeEmail} \\
    \fi
    \ifx
        \tAuthorFourName \empty
            %nothing
        \else
            \LARGE \tAuthorFourName \\
        \large \href{mailto:\tAuthorFourEmail}{\tAuthorFourEmail} \\
    \fi

    \vspace*{\fill}

%============================================================
% Teacher management

    \begin{flushleft}
        \Large \tTeacherOneTitleAndName \\
        \begin{minipage}[t]{0,7\textwidth}
            \large \href{mailto:\tTeacherOneEmail}{\tTeacherOneEmail} \\
            \large \tTeacherOneLocation
        \end{minipage} \\ [0.5cm]

        \ifx
            \tTeacherTwoTitleAndName \empty
                %nothing
            \else
                \Large \tTeacherTwoTitleAndName \\
                \begin{minipage}[t]{0,7\textwidth}
                    \large \href{mailto:\tTeacherTwoEmail}{\tTeacherTwoEmail} \\
                    \large \tTeacherTwoLocation
                \end{minipage} \\ [0.5cm]
        \fi
    \end{flushleft}

%============================================================

    \large \tDate

\end{center}
\end{titlepage}
%\date{}
%\maketitle

% Page style
\thispagestyle{fancy}
\fancyhead[R]{DVA 227 - Projekt i Nätverksteknik}
\fancyhead[L]{M\"{a}lardalen University}
\fancyfoot[L]{}
%\fancyfoot[LE,RO]{\thepage}
\renewcommand{\headrulewidth}{0.4pt}
\renewcommand{\footrulewidth}{0.4pt}

% Begin actual text

\def\abstract{
   \vfil
\begin{center}%
{\bfseries\abstractname\vspace{-.5em}}
\end{center}
\itshape
}

\def\endabstract{\par
}
% ============================= Abstract ==============================
%\begin{abstract}
%To be written
%\end{abstract}
%\newpage
% ========================== Document version ===========================
%\begin{center}
%	\begin{tabular}{|l|l|l|}
 %		
	%	\multicolumn{3}{c}{\textbf{{\large Document version}}} \\
	%	\hline
 	%	Version & Date & Note \\ \hline
 	%	 &  &  \\ \hline
	%	 &  &  \\ \hline
	%	 &  &  \\ \hline
	%	 &  &   \\ \hline
	%	 &  &   \\ \hline
	%	 &  &   \\ \hline
%	\end{tabular}
%\end{center} 
%\newpage
\pagenumbering{roman}
\newpage
%======================== Table of contents ==========================


\hypersetup{linkcolor=black}

%\cleardoublepage
\hypersetup{linkcolor=red}


% ============================== Intro ===============================
\pagenumbering{arabic}
\input{./inledning}

% ============================ Bakgrund ==============================
\section{Periodens Resultat}
\textbf{\LARGE Vecka 16}
\begin{itemize}
\item Gemensamma moment
\begin{itemize}
\item Arbetat på rapporten, revideringar och layoutändringar
\end{itemize}
\item Micke
\begin{itemize}
\item Nerlagd Arbetstid: ca 5.5 timmar
\item Moment utförda
\begin{itemize}
\item Arbetat på rapporten
\end{itemize}
\end{itemize}

\item Ville
\begin{itemize}
\item Nerlagd Arbetstid: ca 5.5 timmar
\item Moment utförda
\begin{itemize}
\item Arbetat på rapporten
\item Skapat ett utkast på postern
\end{itemize}
\end{itemize}

\item Isak
\begin{itemize}
\item Nerlagd Arbetstid: ca 6 timmar
\item Moment utförda
\begin{itemize}
\item Arbetat på rapporten
\end{itemize}
\end{itemize}

\end{itemize}



\section{Pågående aktiviteter}
\begin{itemize}
\item Samla projektet och alla dess filer och skicka in till examinator
\item Förbereda inför mässan
\end{itemize}
\section{Tidsplan \& Budget}
\begin{itemize}
\item Tidsplanen ser fortfarande bra ut
\item Fortfarande inom budget, dvs inga utgifter
\end{itemize}
\section{Kritiska faktorer och åtgärder}
n/a
\section{Prognos}
\begin{itemize}
\item Färdigställa poster och förbereda presentation för mässan
\end{itemize}
\section{Övrigt}
\newpage

% ============================== Title ===============================
%\input{./Name of Tex File}


% ============================= References ============================
\newpage
\bibliographystyle{IEEEtran}
%\bibliography{./references}
%\addcontentsline{toc}{section}{References}
%\addcontentsline{toc}{section}{\numberline{}Referenser}
%\setcounter{section}{3}

% ============================ Appendices =============================
%\newpage


%\begin{appendices}
%\input{./name of appendice.tex}
%\newpage
%\input{./name of appendice 2.tex}
	%\input{./Appendices/appendices1}
	%\clearpage
	
	%\input{./Appendices/appendices2}
	%\clearpage
	
%\end{appendices}

\end{document}

%How to import an image into the text. 
%\begin{figure}[h!]
%	\begin{center}
%		\includegraphics[width=0.40
%        \textwidth]{./name of image.pdf}
%	\end{center}
%		\caption{Caption of image}
%			\label{tcphand} 
%\end{figure}


%Example of a "configuration box" 
-----------------------------------------------------------------



\begin{figure}[h!]
\fbox{\begin{minipage}{\textwidth}
R1\# \textbf{show ip route eigrp}\\
Codes: \hspace{0,1cm} L - local, C - connected, S - static, R - RIP, M - mobile, B - BGP\\
\hspace*{1,3cm} D - EIGRP, EX - EIGRP external, O - OSPF, IA - OSPF inter area\\

\textless Output omitted\textgreater\\
------------------------------------------------------------------------ \\
Gateway of last resort is not set\\

\colorbox{babyblueeyes}{D} \hspace{0,6cm} 172.16.0.0/24 [90/409600] via 10.0.0.2 , 02:32:24, FastEthernet0/1\\
\colorbox{babyblueeyes}{D EX} \hspace{0,1cm} 209.165.201.0/27 [\colorbox{babyblueeyes}{170/537600}] via 10.0.0.2, 02:32:24, FastEthernet0/1\\
\end{minipage}}
\caption{Routingtabell}
                \label{routetable} 
\end{figure}

%Mindre exempel av ovan. 
--------------------------------------------------------
\begin{figure}[h!]
\fbox{\begin{minipage}{\textwidth}
Router(config)\# \textbf{route-map} \textit{RM-NAME} \textbf{permit 10}\\
Router(config-route-map)\# \textbf{match ip address} \textit{ACCESS-LIST-NAME}\\
Router(config-route-map)\# \textbf{set} \textit{VALUE}\\
Router(config)\# !Applicera route-map där den behövs. 

\end{minipage}}
\caption{Generell Route-Map Konfiguration}
                \label{routetable} 
\end{figure}





