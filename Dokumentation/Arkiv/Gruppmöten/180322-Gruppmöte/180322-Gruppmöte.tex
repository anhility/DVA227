% Title
%\title{}

% Document type
\documentclass[11pt, titlepage, a4paper]{article}

% Preamble
% ==================== Packages ====================
% Character Encoding
\usepackage[T1]{fontenc}
\usepackage[utf8]{inputenc}
\usepackage{lmodern}
\usepackage[swedish]{babel} % <-- Change language used here. Rememeber to clean all the aux files before building document.
\usepackage{csquotes}
\usepackage{microtype}

% Bibliography & References Formatting
\usepackage[style=ieee,backend=biber,dateabbrev=false]{biblatex}

% Appendices Formatting
\usepackage[titletoc,title]{appendix}
%\renewcommand\appendixtocname{Bilagor}     % <-- If needed to manually change appendice names
%\renewcommand\appendixpagename{Bilagor}    % <-- If needed to manually change appendice names

% Text Formatting
\usepackage{marginnote}
\usepackage[absolute]{textpos}
\sloppy

% Document Formatting
\usepackage{fancyhdr}
\usepackage{enumerate}
\usepackage{enumitem}
\usepackage{listings}
\renewcommand{\lstlistlistingname}{Listor}   % If needed to manually change list of lists's name in ToC.
\renewcommand{\lstlistingname}{Lista}        % If needed to manually change list's name in document.
\usepackage[margin=3cm]{geometry}
\usepackage[section]{placeins}

% List code style
\lstset{
    backgroundcolor=\color[rgb]{0.92,0.92,0.92},
    basicstyle=\ttfamily,
    columns=fullflexible,
    keepspaces=true,
    frame=single,
    showspaces=false,
    showstringspaces=false,
    showtabs=false,
    tabsize=2,
    captionpos=b,
    breaklines=false,
    keywordstyle=\color[rgb]{0,0,1},
    commentstyle=\color[rgb]{0.133,0.545,0.133}
}

% Urls & Hyperref
\usepackage{url}
\usepackage[pdfborder={0 0 0}, colorlinks=true, urlcolor=blue, citecolor=red, bookmarks=false]{hyperref}

% Image and Table Formatting
\usepackage{tabularx}
\usepackage{caption}
\usepackage{graphicx}
\usepackage{array}
\usepackage{diagbox}
\usepackage{chngcntr}
\usepackage{multirow}
\usepackage{color, colortbl}

% Tikz
\usepackage{tikz}

% Math
\usepackage{amsmath}
\usepackage{amsfonts}
\usepackage{mathtools}
\usepackage{bm}

% Other
\usepackage{lipsum}
\usepackage{datetime}
\bibliography{static/biblio}

%============================== Custom macros for document

%=============== Front page info
%% Logo info
\def\tLogoLocation{static/MDHlogga.png}
\def\tLogoWidth{15mm}

%% School info
\def\tSchoolName{Mälardalen University}
\def\tAcademyName{School of Innovation, Design and Engineering}     % IDT
%\def\tAcademyName{School of Education, Culture and Communication}   % UKK
\def\tSchoolLocation{Västerås, Sweden}

%% Course info
% Leave the fields for titles, authors and teachers blank if you don't need them.
% e.g. \tTitleThree{}
% They will automatically be hidden on the front page.

\def\tCourseName{DVA227 - Projekt i Nätverksteknik}
\def\tTitleOne{Projekt Bolaget}
\def\tTitleTwo{Mötesprotokoll}

\def\tTitleThree{Gruppmöte - 22/3-18}
%\def\tTitleThree{Projektledarmöte}

%% Author info
\def\tAuthorTitle{Grupp 4}     % <-- Set to {} to hide

\def\tAuthorOneName{Mikael Andersson \textit{- projektledare}}       % <-- Set to {} to hide
\def\tAuthorOneEmail{man16057@student.mdh.se}
%======
\def\tAuthorTwoName{Vilhelm Beijer}       % <-- Set to {} to hide
\def\tAuthorTwoEmail{vbr16001@student.mdh.se}
%======
\def\tAuthorThreeName{Isak Söderström}      % <-- Set to {} to hide
\def\tAuthorThreeEmail{ism16001@student.mdh.se}
%======
\def\tAuthorFourName{}    % <-- Set to {} to hide
\def\tAuthorFourEmail{}

%% Teacher info
\def\tTeacherOneTitleAndName{Kursansvarig, Lärare \& Examinator: Joakim Rydén}
\def\tTeacherOneEmail{joakim.ryden@mdh.se}
\def\tTeacherOneLocation{\tSchoolName, \tSchoolLocation}
%======
\def\tTeacherTwoTitleAndName{Lärare: Sara Lundahl}  % <-- Set to {} to hide
\def\tTeacherTwoEmail{sara.lundahl@mdh.se}
\def\tTeacherTwoLocation{\tSchoolName, \tSchoolLocation}

%% Misc
\def\tDateDocMade{22 mars 2018}     % Use \today or set a date manually

\def\tDates{\begin{tabular}{rl}
        Skapad: & \tDateDocMade \\
        Uppdaterad: & \today\,\,\currenttime
    \end{tabular}
}

%=============== Document Headers
\def\mLHeader{\tSchoolName}
\def\mRHeader{\tTitleTwo\ - \tTitleThree}
%============================== End of macros

% Page style
\pagestyle{fancy}
\marginparsep = 10pt
\renewcommand*{\marginfont}{\footnotesize}

% Table of Contents
\renewcommand\contentsname{Table of Contents}
\newcommand{\HRule}{\rule{\linewidth}{0.5mm}}
\newcommand{\circR}{\textsuperscript{\textregistered}}

% Document begins
\begin{document}

    % Title
    \hypersetup{pageanchor=false}
    \begin{titlepage}
        \begin{center}
    \begin{figure}[t]
        \includegraphics[width=\tLogoWidth, viewport=0 0 100 100]{\tLogoLocation}
    \end{figure}

    \Large \tSchoolName \\
    \Large \tAcademyName \\
    \Large \tSchoolLocation \\

    \noindent\makebox[\linewidth]{\rule{\textwidth}{0.4pt}} \\ [0.5cm]

    \Large{\tCourseName} \\ [1.0cm]

%============================================================
% Title management

    \Huge \textbf{\underline{\tTitleOne}} \\ [0.5cm]
    \ifx
        \tTitleTwo \empty
            %nothing
        \else
            \huge \textbf{\uppercase{\tTitleTwo}} \\
    \fi
    \ifx
        \tTitleThree \empty
            ~\\ [0.5cm]
        \else
            \huge \textbf{\uppercase{\tTitleThree}} \\ [1cm]
    \fi

%============================================================
% Author management

    \ifx
        \tAuthorTitle \empty
            %nothing
        \else
            \LARGE \textbf{\underline{\smash{\tAuthorTitle}}} \\ [0.2cm]
    \fi
    
    
    \LARGE \tAuthorOneName \\
    \large \href{mailto:\tAuthorOneEmail}{\tAuthorOneEmail} \\
    
    \ifx
        \tAuthorTwoName \empty
            %nothing
        \else
            \LARGE \tAuthorTwoName \\
            \large \href{mailto:\tAuthorTwoEmail}{\tAuthorTwoEmail} \\
    \fi
    \ifx
        \tAuthorThreeName \empty
            %nothing
        \else
            \LARGE \tAuthorThreeName \\
            \large \href{mailto:\tAuthorThreeEmail}{\tAuthorThreeEmail} \\
    \fi
    \ifx
        \tAuthorFourName \empty
            %nothing
        \else
            \LARGE \tAuthorFourName \\
        \large \href{mailto:\tAuthorFourEmail}{\tAuthorFourEmail} \\
    \fi

    \vspace*{\fill}

%============================================================
% Teacher management

    \begin{flushleft}
        \Large \tTeacherOneTitleAndName \\
        \begin{minipage}[t]{0,7\textwidth}
            \large \href{mailto:\tTeacherOneEmail}{\tTeacherOneEmail} \\
            \large \tTeacherOneLocation
        \end{minipage} \\ [0.5cm]

        \ifx
            \tTeacherTwoTitleAndName \empty
                %nothing
            \else
                \Large \tTeacherTwoTitleAndName \\
                \begin{minipage}[t]{0,7\textwidth}
                    \large \href{mailto:\tTeacherTwoEmail}{\tTeacherTwoEmail} \\
                    \large \tTeacherTwoLocation
                \end{minipage} \\ [0.5cm]
        \fi
    \end{flushleft}

%============================================================

    \large \tDate

\end{center}
    \end{titlepage}
    \hypersetup{pageanchor=true}

    % Page style
    \pagestyle{fancy}
    \fancyhead[L]{\mLHeader}
    \fancyhead[R]{\mRHeader}
    \setlength{\headheight}{14pt}
    \fancyfoot[L]{}
    \fancyfoot[R]{}
    \renewcommand{\headrulewidth}{0.4pt}
    \renewcommand{\footrulewidth}{0.4pt}

    % Reset of counters after each section
    \counterwithin{figure}{section}
    \counterwithin{table}{section}
    \counterwithin{lstlisting}{section}

%========================== Table of contents =========================
    \pagenumbering{Roman}
    \hypersetup{linkcolor=black}

    % Table of contents
    %\tableofcontents        % <-- Comment out if not needed
    %\newpage

    % List of figures
    %\listoffigures          % <-- Comment out if not needed

    % List of tables
    %\listoftables           % <-- Comment out if not needed

    % List of lists
    %\lstlistoflistings      % <-- Comment out if not needed

% =========================== Actual Content ==========================
    \newpage
    \pagenumbering{arabic}
% ========== Add your texts here

    % Kommentera bort de punkter som inte behövs.

\section*{Närvarande}
\begin{itemize}[noitemsep]
    \item Mikael
    \item Isak
    \item Vilhelm \textit{- sekreterare}
    \item Joakim
    \item Sara
\end{itemize}

\section*{Genomgång av föregående mötesprotokoll}
\begin{itemize}[noitemsep]
    \item 180322-Gruppmöte-[beslut].pdf
\end{itemize}


\section*{Var har gjorts sedan föregående möte?}
\begin{itemize}[noitemsep]
    \item Förstudien
    \begin{itemize}[noitemsep]
        \item Preliminär WBS
        \item Preliminär MoSCoW-analys
        \item QTR-analys
        \item Preliminär logisk nätplan
    \end{itemize}
    \item Gruppkontrakt renskriven och signerad.
\end{itemize}

\subsection*{Problem/Åtgärder?}
n/a

\section*{Vad ska göras innan nästa möte?}
\begin{itemize}[noitemsep]
    \item Avsluta förstudien
    \item Starta planeringsfasen
    \item Planering
    \begin{itemize}[noitemsep]
        \item Fastställa slutgiltig WBS.
        \item Fastställa slutgiltig MoSCoW-analys.
        \item Fastställa slutgiltig logisk nätplan.
        \item Börja med riskanalys.
    \end{itemize}
\end{itemize}

\subsection*{Problem/Åtgärder?}


\section*{Beslut}

\section*{Uppföljning}

\section*{Övrigt}
\begin{itemize}[noitemsep]
    \item Bestämma grupparbeten för v14.
    \item Bestämma gruppmöte för v14.
    \item Ladda upp dagbok senast kl 21:00 på torsdag 29/3-18 så projektledare kan skriva progressrapport för v13 innan fredag då han kommer vara borta under helgen.
\end{itemize}

% ============================= References ============================
% Comment out these lines if you don't need a list of references.
    %\clearpage
    %\pagenumbering{roman}
    %\printbibliography[heading=bibintoc]

% ============================= Appendices ============================
% Comment out these lines if you don't need a any appendices
    %\clearpage
    %\begin{appendices}
       %\section{Första Bilagan}
En enkel bilaga. Varje sektion skapar en ny bilaga i dokumentet.

\subsection{En subsektion i bilagan}
Exakt vad det står. Blir en del av den redan skapade bilagan.

\section{En annan sektionstitel i första bilagan}
Skapar en ny bilaga.
       %\section{En ny bilaga fil}
Inget mer att se här.
    %\end{appendices}

% ============================ Document End ===========================
\end{document}
