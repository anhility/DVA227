% Kommentera bort de punkter som inte behövs.

\section*{Närvarande}
\begin{itemize}[noitemsep]
    \item Mikael
    \item Isak \textit{- sekreterare}
    \item Ville
\end{itemize}

\section*{Genomgång av föregående mötesprotokoll}
\begin{itemize}[noitemsep]
    \item 180228-Gruppmöte-[beslut].pdf
\end{itemize}

\section*{Var har gjorts sedan föregående möte?}

\begin{itemize}[noitemsep]
    \item Samarbetsyta på github skapad och använd.
    \item Protokollmallar skapade och redo att användas.
    \begin{itemize}[noitemsep]
        \item YYMMDD-rapporttyp-[taggar].pdf
    \end{itemize}
    \item Projektledarordning vald.
    \item Projektspecifikation skriven.
    \item Dagböcker inskaffade och utdelade.
    \item Email till Johnny om att projektet har startat.
    \item Projektledar träff 1 avklarad.
    \item Förstudie startad.
    \begin{itemize}[noitemsep]
        \item Första WBS skriven och dokumenterad.
        \item Första MoSCoW skriven och dokumenterad.
    \end{itemize}
\end{itemize}

\subsection*{Problem/Åtgärder?}
\begin{itemize}[noitemsep]
    \item Möjligtvis byte fån github till annan samarbetsyta.
    \begin{itemize}[noitemsep]
        \item (M) förslag: Testa med github i ca 2 veckor innan vi ändrar samarbetsyta.
    \end{itemize}
\end{itemize}

\section*{Vad ska göras innan nästa möte?}
\begin{itemize}[noitemsep]
	\item Göra klart WBS och MoSCoW
	\item Göra en QTR triangel
	\item Göra en logisk nätplan
    \item Bli klar med förstudien.
    \item (M): Planera projektstrukturen.
    \item (M): Planera planeringsfasen.
    \item Starta planeringsfasen

\end{itemize}

\subsection*{Problem/Åtgärder?}

\section*{Beslut}
\begin{itemize}
	\item Github ska prövas i 2 veckor innan nytt beslut ska ske om vi ska byta sammarbetsyta.
	\item Grupparbete vecka 13, Måndag 09:00-15:00, Torsdag 09:00-12:00.
\end{itemize}
\section*{Uppföljning}

\section*{Övrigt}